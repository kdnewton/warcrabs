\documentclass[a4paper,10pt]{article}
\usepackage{textcomp}
\usepackage{amsmath}
\usepackage{hyperref}

\hypersetup{colorlinks=true,urlcolor=red}

\newcommand{\warcrabs}{\textbf{Warcrabs!}}
\newcommand{\inch}{\textacutedbl\hspace*{4pt}}
\newcommand{\dimoEmail}{\href{mailto:dimolaoshi-AT-yahoo~dot~com}{dimolaoshi [[nospam AT]] yahoo [[nospam .]] com}}
\newcommand{\kyleEmail}{\href{mailto:kyle-AT-n3wt0n~dot~com}{kyle [[nospam AT]] n3wt0n [[nospam .]] com}}

%opening
\title{Warcrabs! Manual}
\author{\small{Rules and concept by Di Mo. Revision by kdnewton.}}

\begin{document}

\date{}
\maketitle{}

\begin{center}
 \small{Version 1+}
\end{center}

\begin{quote}
Death Boy lifted his claw and gently pushed aside the lush green grass in front of him. His swiveling eye-stalks took in the terrain. The desk was piled edge to edge with TPS reports. Two crumpled Skittles wrappers rested, like tumbleweeds, near the computer monitor. On the far side of the computer, he could make out an inbox stuffed with assorted papers, and nearby, a grotty, unwashed coffee cup.

Where was his enemy? Clearly, there were only a few places a crab could be hiding. Death Boy already had one of the best hiding places -- inside the Chia pet. But he had done this before, and his enemy would be looking for him here. He had to make a move.

Death Boy clambered out of the grass and swung down to the desktop. He hefted his laser pointer and eyed the computer monitor. He would have to make a dash for it -- running, and if need be, firing.

Death Boy skittered across the desktop, claws clattering like manicured nails. His eye-stalks peered out over his raised laser pointer, ready to obliterate anything that moved.

And something did move. As he neared the monitor, one of the Skittles wrappers rustled ever so slightly. With a tiny squeal of war-passion, Death Boy fired into the wrapper, searing hole through it. He charged directly for the wrapper, holstering the laser pointer and drawing his pickle spear. He would finish this off claw-to-claw, like a true crab.

Death Boy ripped aside the smoldering Skittles wrapper and got his first look at the enemy. A rookie, barely past his first molt, sprawled dying on the desktop. Death Boy flipped the crab on his back, ready to insert the spear for the killing blow. Only then did he see his fatal mistake.

That black object in the rookie's hand was not a laser pointer. It was a firecracker! The fuse sprang to life, suddenly illuminating the scene with terrible clarity. Death Boy saw the black cats on the side of the firework. He saw the weary, too-old look in the eye-stalk of the wounded rookie.

It was the last thing Death Boy saw before he and the rookie died in an explosion of blood and chitin....
\end{quote}

\section*{Introduction}
\label{sec:Introduction}

The future is bright at Associated Holistic Technologies. When the company was founded in 1997, venture capitalists were making a killing on Internet startups. But it took an old bricks-and-mortar guy -- legendary Texas oilman T. Fred Hawkins -- to see the real future of consumer technology. Declaring that ``the next century is a biotech century,'' Hawkins started AssHoTech with the dream of building it into the ``Microsoft of genetic engineering.''

Ten years later, AssHoTech has indeed grown into a biotech giant. It's the nation's leading producer of baboon stem cells. Its florescent fruit flies are the gold standard for research. Its work for Homeland Security, though highly classified, is reputed to be of vital interest to the security of the United States -- and it had better be, given the size AssHoTech's contracts. Small wonder that Business Week labeled AssHoTech ``the place where the Nineties never ended.'' For AssHoTech workers, it's still possible to become a millionaire before thirty, to land hot chicks by bragging about your stock options, and to wear a Hawaiian shirt in the office. 

It's also a mind-numbingly boring place to work. AssHoTech's numbers may cause the business journalists to swoon -- but the company's 4,000-acre Alpharetta campus is a windowless maze of 21st-Century ennui. Biotech, as it turns out, is all about running the same experiment over and over again. You may have been celebrated as a wunderkind while getting your Ph. D., but now you spend your days cleaning Petri dishes and maintaining equipment like a graduate assistant. You may have masterminded the bold ad campaign that brought CheapJet back from ruin, but here your job is to keep it bland and non-controversial while the big checks roll in. You may have fled here from your birth country in the wheel well of an Airbus, but now that you have heard ``The Girl From Ipanema'' for the millionth time over the company speaker system, you are wondering if you should have ever left.

For years, AssHoTech was a primordial soup of overeducated, underemployed brainpower. So it is perhaps not surprising that new life forms emerged. As wireless technology and unpaid overtime erased the distinction between home and work, AssHoTech's lab workers and engineers began bringing their own hobby projects into the lab. First they made Chia Pot -- tiny farms of dwarf marijuana that anybody could grow on a desktop. Then they made Z-Monkeys, human-looking brine shrimp with monster-sized libidos. But things didn't really get crazy until an AssHoTech biologist brought in a box of common hermit crabs -- the kind you find in a pet store -- for a few evil experiments.

Placed in a too-crowded tank, hermit crabs will often fight with one another, but this fighting is too slow-paced and sporadic to be of any real interest. So our bored biologist altered the crabs, making them faster, smarter, and more aggressive.

Within weeks, crab-fighting groups had sprung up across the sprawling AssHoTech campus. Crab-fighting held something for everybody. Biologists went on tweaking genes to make crabs better and more lethal. Because the crabs would live in just about any shell-like object, the IT and engineering guys began building armored, weapon-laden shells for their little warriors. Business and marketing types became bookies in the complex betting network that grew around the fights, and public relations officers -- so useless in the real world -- used their psychological warfare skills and devious natures to become lords of the crab-fighting arena.

Morale was never better at AssHoTech. Productivity was up. The company issued a flurry of patent requests as its biotech guys, tweaking their hobby projects, overcame fundamental hurdles and streamlined the gene-splicing process. The company's website and press releases seemed to have an added panache. It was as if, suddenly, someone was home and the lights were on. One middle manager did an informal study, and concluded that a worker who wasted an hour of company time per day on crab-fighting would actually be more productive than the worker who put in a full eight hours.

But it couldn't last. For one thing, there was no way for the company to make crab-fighting semi-official. CEO T. Fred Hawkins was well-known for making chummy visits to low-level employees, taking them out for beers and listening to their observations. Early in the emergence of crab-fighting, workers concluded that high-level administrators would never be allowed in the ring. All of AssHoTech was soaked in Hawkins' pushy charisma, and the workers wanted crab-fighting to be a charisma-free zone.

And then there were the legal and public relations ramifications. Animal activists were already upset about the baboons who were dying in agony on B wing. The general public was fairly queasy about those human organs growing in pigs at the AssHoTech Ranch in Cordele. The FDA and EPA were expressing concern about cross-pollination and the prospect of modified animals accidentally getting into the food chain. The American public might be okay with Frankenfood, but they weren't going to accept massive genetic manipulation of an innocent animal when it didn't result in some sort of marketable product.

So they banned it. On March 20, 2005, T. Fred Hawkins issued a lengthy memo now known, in AssHoTech circles, as the Riot Act. Though it never mentioned crab-fighting by name, or even acknowledged a problem, the memo laid out harsh penalties for anyone caught ``working on any non-approved project on company time, or with company equipment'' or ``bringing the product of any said work into the workplace.'' The penalty for said work was immediate termination. The memo did not stop crab-fighting, but it did send it underground.

Despite the ban -- or perhaps because of it -- one of AssHoTech's worst fears was realized. Less than a year after the Riot Act was issued, a custodian found a colony -- indeed, a tiny city -- of wild warcrabs living in the sub-basement of D Wing. Crabs were on the loose in the building. They were aggressive. And they were very, very smart....

\section*{How to Play}
\label{sec:HowtoPlay}

\warcrabs\hspace*{4pt}is a simple wargame designed to be played in the workplace. In the game, you take on the role of an AssHoTech worker, creating and managing one or more crabs, collecting the spoils of victory, and attempting to become the best crab-fighter in the building.

This is a true pen-and-paper game. To play, you will need to print a Crab Character Sheet and record the data about your crab on it. But there are no maps, and no figures. The playing field is your own home or office -- where a simple desktop can become an arena of courage and death for a few genetically-altered crustaceans. To begin playing, you will need a few common household implements.

\begin{itemize}
 \item A paperweight, or similar small object, to represent your crab. The object must have an obvious ``front'' and ``back'' side.
 \item A ruler, or better, a tape measure.
 \item A pair of six-sided dice (you've got a Monopoly set around here somewhere, don't you?)
\end{itemize}

Through the course of the game, you will move your crab (paperweight) around the battlefield (desktop) using the ruler/tape measure to determine how far you will be able to move or shoot during each turn.

\section*{Engineering Your Crab}
\label{sec:EngineeringYourCrab}

So you're a new worker at AssHoTech, and one of your co-workers has surreptitiously invited you to take part in a crab-fighting ring. At this point, you would head to the lab to start splicing genes -- or, if you're not a biotech guy, you would purchase a crab from someone who can engineer them.

To simulate the crab engineering process, \warcrabs\hspace*{4pt}uses a point-allocation system to help you decide which attributes your crab should possess. To start creating your crab, whip out a copy of the Crab Character sheet.

Warcrabs have lots of neat traits (they can be trained to smoke, for instance, and they enjoy martial arts movies), but for purposes of combat, only three are really important.

\begin{description}
 \item[Strength --] Determines how much your crab will be able to carry. Once the crab acquires an armored, weapon-laden shell, its Strength score will also affect its speed of movement.

 \item[Dexterity --] Determines how well a crab can shoot and how fast it can move.

 \item[Endurance --] Determines how much damage a crab can take before it dies.
 \end{description}

Each trait is rated on a scale of 1 to 10 (a crab with a Strength 10 is as strong as he can be, and a crab with a Strength of 1 is very weak).

In designing your crab, you have 15 points to distribute among the three traits. So you could create a crab with a Strength of 5, a Dexterity of 5 and an Endurance of 5, or you could try a different combination, such as a strong crab (Strength 8) that isn't very dexterous (Dexterity 2).

\begin{description}
 \item[Shells --] In nature, hermit crabs don't grow shells of their own: they live in shells that are abandoned by other animals, moving from one shell to another as they grow larger. One might say that these animals are primitive users of technology.
 \end{description}

There's nothing primitive about the technology used in \warcrabs. Accomplished fighting crabs are decked out with high-tech, armored shells, featuring a wide array of weapons systems, all created (or miniaturized) by the techies at AssHoTech.

Each crab starts the game with a simple shell of the sort you might find on the beach. Even this basic, low-tech shell provides some armor to protect the crab from attacks from the side, and the rear. The Basic shell provides 10 points of armor, which means it can absorb 10 points of damage.

To find out how much equipment your crab can eventually hold, you need to determine your Optimum Shell Capacity and your Maximum Shell Capacity. Optimum Shell Capacity is the amount of weight your crab will be able to hold without any loss to its speed of movement. Maximum Shell Capacity is absolutely the largest amount of equipment you can cram into a shell. (A maxed-out crab might have a lot of equipment but would essentially be a fortress, unable to move: not a good thing in combat.)

\begin{description}
 \item[Optimum Shell Capacity --] The same as its Strength rating.
 \item[Maximum Shell Capacity --] same as its Strength plus its Dexterity.
 \item[\normalfont{Example:}] Death Boy has a strength of 6 and a Dexterity of 4. His Optimum Shell Capacity is 6 and his Maximum Shell Capacity is 10.

 \item[Weight --] Weight is a measure of the amount of equipment the crab is carrying right now. This number can never exceed the crab's Maximum Shell Capacity. Unfortunately, even a Basic shell carries some weight -- so mark your weight down as 1.

 \item[Armor --] As mentioned before, the Basic shell carries 10 points of armor, so mark your Armor number as 10.

 \item[Hit Points --] The crab's Hit Points show how much damage it can take before being killed. Your crab's Hit Points are the same as its Endurance.

 \item[Movement --] Your movement number is a measure of how many inches your crab can move in a single turn. A crab's basic Movement number is the same as his Dexterity number. However, when a crab is carrying too much weight, his movement is slowed. If your Weight is higher than your Optimum Shell Capacity, determine the difference between the two numbers and subtract that number from your Movement.
 \item[\normalfont{Example:}] Death Boy has an Optimum Shell Capacity of 6 and a Dexterity of 4. When he starts the game, he has only a Basic shell with one point of Weight, so his Movement is 4 until he exceeds his Optimum Shell Capacity. Later, he acquires 7 Weight points of equipment. Because his shell weighs more than his Optimum Shell Capacity he determines the difference (1 point) and subtracts that from his Movement. His Movement is now at 3, and he can move only 3 inches per turn.

 \item[Money --] You'll need money to start buying equipment. Folks at AssHoTech make a lot of dough, but to keep the fight fair, crab-fighters are allowed to spend only the prize money they've won in various crab-fights. By long-standing crab-fighting custom, all crab-fighters are expected to begin with \$1.98 in their accounts (an homage to Chuck Barris, perhaps). So write \$1.98 on your crab sheet's Money blank for now.
 \end{description}

\section*{Combat}
\label{sec:Combat}
\subsection*{Turns}
\label{sec:Turns}

Combat in \warcrabs\hspace*{4pt}is organized in turns. One turn in the game represents one second in real time.

In each turn, a crab is allowed to move, and to take one action (usually firing a weapon or some other attack). The crab with the highest Dexterity gets to move and shoot first. If two crabs have the same Dexterity, the players roll the dice -- the player with the highest number gets to move first.

\subsection*{Rolls}
\label{sec:Rolls}
If a crab tries to perform a skill that involves one of the three traits, the player should roll two six-sided dice. If the number rolled is less than or equal to the crab's score in that trait, the crab succeeds in his task. This is called ``rolling against'' a trait.

In combat, crabs must roll against Dexterity to determine whether they can hit an opponent with a weapon. This is probably the skill crabs will roll against most often.

\begin{description}
 \item[\normalfont{Example:}] Death Boy has a Strength of 6 and a Dexterity of 4. His Opponent, the Crustaceanator, has an Endurance of 3. In combat, Death Boy tosses a firecracker at the Crustaceanator. He rolls against Dexterity and gets a 5 -- meaning that he just missed. However, because the firecracker landed close to the Crustaceanator, the Crustaceanator must roll against Endurance or else be stunned for one turn. The Crustaceanator rolls an 8, meaning that he is stunned. Death Boy runs over to the Crustaceanator and tries to lift the crab over his head to fling him off the desktop. Death Boy rolls a 4 and the Crustaceanator plunges to his death on the office floor below.
 \end{description}

\subsection*{Modifiers}
\label{sec:Modifiers}

All tasks are not created the same. It's easy to hit an opponent who is standing right next to you. It's very hard to hit an opponent who is far away. Thus we have ``modifiers'' that reflect that difficulty in the game world.

The most important kind of modifier is the ``To Hit'' modifier. It gets harder to hit a target as the target gets farther away. To determine the To Hit modifier for any shot, just measure the distance between the shooter and the target, then compare to the table below:

\begin{center}
% use packages: array
\begin{tabular}{ | l | l | r | }
\hline
\textbf{Range} & \textbf{Distance} & \textbf{Modifier} \\ \hline
Point Blank & Less than 2\inch & +2 \\ \hline
Short & 2\inch to 6\inch & +1 \\ \hline
Medium & 6\inch to 12\inch & 0 \\ \hline
Long & 12\inch to 18\inch & \textminus1 \\ \hline
Extra Long & 18\inch or more & \textminus2 \\
\hline
\end{tabular}
\end{center}

You'll find that various weapons also come with built in To Hit Modifiers. For instance, a Bottle Rocket Launcher has a To Hit Modifier of \textminus1 (to reflect the perils of using an unguided rocket system.) However, the MBRLS (Multiple Bottle Rocket Launcher System) fires a swarm of rockets at once, increasing the chance to hit the target. It has a To Hit modifier of +1.

To determine your chance of hitting at a certain range, add all To Hit modifiers to your crab's dexterity and roll against that number.

\begin{description}
 \item[Claw-to-Claw Combat:] In the beginning, crabs may not have enough money to buy lots of firecrackers, laser pointers and other weapons. They will probably engage in Claw-to-Claw combat, using simple weapons like Pickle Spears. There are no range-based To Hit modifiers for Claw-to-Claw combat. You must be within 1 inch of the target to begin Claw-to-Claw, and if you are using a Claw-to-Claw weapon, you just roll against your Dexterity to hit. If you're carrying a ranged weapon, you get the +2 modifier for being at Point Blank range. (A good reason not to charge an opponent who is carrying a ranged weapon.)

You can even attack with your \textbf{bare claws}, which are good for snapping off an opponent's limbs (they will grow back.) A bare-claw attack will do a single point of damage if it hits.

 \item[Armor --] The armored shell definitely adds a wrinkle to crab combat. To inflict damage and kill your opponent, you have to hit the ``meaty'' part of the crab -- but that can be quite difficult.

 \item[Attacks from behind --] Crabs are almost invulnerable from behind. Any attack from behind a crab, if it hits, will damage only the crab's shell. You'll have to blast all the way through the crab's shell to get to the crab itself. For game purposes, ``behind'' the crab is anywhere within the 180\textdegree\hspace{4pt}zone opposite the crab's front side.

 \item[Attacks from the side --] For game purposes, you are attacking from the side if you are anywhere in the zone that extends from 90\textdegree\hspace{4pt}to 45\textdegree\hspace{4pt}from the crab's front side. If you are attacking from the side, you have a fifty/fifty chance of hitting crabmeat. If you fire from the side and hit, roll one die. If you roll a 1, 2 or 3, the damage will be applied to your opponent's shell. If you roll a 4, 5 or 6, the damage will be applied to the crab itself.

 \item[Attacks from the front --] If you are within 45\textdegree\hspace{4pt}of the front side of your opponent, you strike crabmeat every time you hit that opponent.

 \item[Using Common Sense --] Now, you could get out a protractor every time you launch an attack, but it's usually pretty easy to tell a 45\textdegree\hspace{4pt}angle just by eyeballing it. This game is supposed to be quick and easy, so it may be best to take a sporting attitude and guess at the angles, pulling out a protractor only when there is a heated argument. It's up to you.

 \item[Explosives --] Explosives are fun. In most wargames, explosives are also very complicated, necessitating a lot of tables dealing with blast radius and so forth. In \warcrabs, there are just three explosive rules.

 \begin{enumerate}
  \item If you fire or toss an explosive at an opponent and miss your roll by \textit{only one point}, the explosive lands close enough to shell-shock your opponent. That opponent must roll against endurance or be stunned for an entire turn.
  \item If you use an explosive in Claw-to-Claw combat (usually a kamikaze move by a crab on the verge of death), both crabs in the combat take the full damage of the explosion to their crabmeat -- in other words, they lose Hit Points, not armor.
  \item If you are hit by an explosive, you must move your crab \textit{one inch farther} from the opponent who fired that explosive. This simulates the blast effect. In other words, if an opponent 10 inches away hits you with a Bottle Rocket, you must move your crab so that you are now 11 inches away from him.
 \end{enumerate}
 \item[Falls and jumping --] Because crab-fighting is often done on desktops, there is the real possibility that your crab will be blasted or thrown off the side of the desk. The rules for blast effects are listed above.

 A crab can throw another crab if the other crab is stunned, and if he passes a Strength check. To find out how far, in inches, a crab can throw another crab, subtract the throwing crab's weight from his Strength.

 A crab takes one point of damage for every foot he falls (falls less than one foot do no damage). Also, a crab who falls must roll against Endurance, or be stunned for one turn. Most desks are about three feet above the ground.

 Crabs can continue fighting after a crab is thrown or falls off a desk or other surface. However, the crab shooting from above has a +1 modifier to hit (in addition to other modifiers) and the crab shooting from below has a \textminus1 modifier to hit.

 A crab, standing flat-footed, can jump upward a number of inches equal to his Movement score. If a crab is moving and jumping, the lateral movement and the upward movement cannot exceed his Movement score.

 \begin{description}
  \item[\normalfont{Example:}] Death Boy has a Movement score of 4, so he can move 2 inches and jump upward 2 inches in a single turn.
  \end{description}

 \item [Cover --] Obviously, hiding behind stuff is a good way to avoid being hit. A crab who is behind a substantial object (coffee cup, inbox, etc), and is out of the line of sight of his opponent, is considered shielded from attacks by most weapons (for the exceptions, read individual weapon descriptions in the Equipment section). Crumpled paper, tissue paper, or old candy wrappers can act as a one-time shield against flamethrower attacks (the first blast strikes the paper; by the next turn, the paper is effectively gone.)
\end{description}

\section*{Equipment}
\label{sec:Equipment}
Here are the equipment items you can buy for your crab. Each equipment name is followed by a list of statistics in the following format: (\textit{Cost, damage, weight, modifiers}).

If you're new to wargaming, please note that damage is annotated in terms like ``1D6,'' ``2D6'' and so on. This is gamers' code: 1D6 means that you roll one six-sided die to determine the weapon's damage, 2D6 means you roll two dice and so forth.

Modifiers include any factors, peculiar to the weapon, that affect firing or damage, including the ``To Hit'' modifier and the weapon's maximum range.

\subsection*{Weapons}
\label{sec:Weapons}

\paragraph*{Pickle Spear}
\label{sec:PickleSpear}
\hspace*{0px}\\(\textit{\$0.20, 1D6, Weight 1, maximum range 1"})\\
The standard weapon for Claw-to-Claw combat, the pickle spear is any small toothpick-like object that can be used by one crab to stab another. Pickle spears come in two main forms -- a fancy toothpick of the sort used to hold a hamburger together, and a tiny plastic sword of the sort you sometimes get in a mixed drink. A crab can create an improvised Pickle Spear in combat, if he can grab a stray paper-clip. To turn the paper-clip into a weapon, the crab must spend one turn bending the clip and must successfully roll against Strength.

\paragraph*{Vorpal Pickle Spear}
\label{sec:VorpalPickleSpear}
\hspace*{0px}\\(\textit{\$1.00, 1D6 +2, Weight 1, maximum range 1"})\\
A metal version of the pickle spear, charged with electricity. Expensive and heavy, the vorpal pickle spear adds only +2 to the damage caused by a normal pickle spear. However, when turned on it is shrouded in a neat, Kirbyesque field of crackling energy. For some players, that effect alone makes it worth using.

\paragraph*{Pointer}
\label{sec:Pointer}
\hspace*{0px}\\(\textit{\$2.00, 1D6, Weight 1})\\
An old standby in crab-fighting, the pointer is a tiny laser pistol/rifle fashioned from a key-chain laser pointer. (The transformation from simple pointer to High-Energy Laser was done by the guys in AssHoTech's Defense Contracts Division, so we are not allowed to explain how this was done.) The pointer is a simple point-and-shoot weapon, with effectively unlimited charges. It can burn through paper/wrappers instantaneously: it will burn through any other cover at a rate of \textonequarter\inch per second -- so an opponent hiding behind a picture frame, slightly less than \textonequarter\inch thick, would get one turn of protection from the pointer.

\paragraph*{Flamethrower}
\label{sec:Flamethrower}
\hspace*{0px}\\(\textit{\$2.00, 2D6, Weight 2, To Hit +2, maximum range 10"})\\
Re-engineered from a common fireplace lighter, the flamethrower is a bulky, hazardous and satisfyingly evil weapon that shoots a stream of fire approximately 10 inches from the shooter. Because it casts a wide swath of flame, the flamethrower carries a +2 To Hit modifier. However, it only carries enough fuel for 10 shots. If a crab's shell armor is completely destroyed while carrying a flamethrower, he will go up in a ball of flame the next time he is hit by a ranged weapon and will lose one Hit Point per turn until he dies.

\paragraph*{Firecracker}
\label{sec:Firecracker}
\hspace*{0px}\\(\textit{\$0.25 each, 2D6, Weight 1 each, maximum range = Strength \texttimes 2"})\\
The hand grenade of \warcrabs. Light it, toss it, watch it explode. Each firecracker takes up one Weight point, but after they're tossed, the weight penalty is gone. The range is limited by your crab's Strength.

\paragraph*{Bottle Rocket}
\label{sec:BottleRocket}
\hspace*{0px}\\(\textit{\$0.30 each, 2D6, Weight 1 each, To Hit \textminus1})\\
Ever break the long red stick off a bottle rocket and watch it skitter across the pavement? Then you have some sense of what this weapon is about. Somewhat accurate at the firing distances involved in \warcrabs, the Bottle Rocket is usually fired from a disposable launcher made from a drinking straw. As with firecrackers, the bottle rocket is discarded after use.

\paragraph*{MBRLS (Multiple Bottle-Rocket Launching System)}
\label{sec:MBRLS}
\hspace*{0px}\\(\textit{\$2.25, 3D6, Weight 5, To Hit +2})\\
Five bottle rockets, taped together, and fired at once. Unless you are really clumsy, one or more is bound to hit something. Can be used only once. The MBRLS is discarded after use.

\paragraph*{Microwave Gun}
\label{sec:MicrowaveGun}
\hspace*{0px}\\(\textit{\$2.50, 2D6, Weight 3, special modifiers below})\\
Inspired by a weapon used in an old Judge Dredd comic, the microwave gun fires a high-intensity beam that can cook crabmeat in its shell. The microwave gun can fire right through any cover, and through a crab's shell, to do 1D6 damage directly to the crab's Hit Points. However, the microwave beam must be set for a certain specific range -- say, 10 inches. It takes one turn to set the range to a new setting, but a crab may move to a position where the current range will work.

\begin{description}
 \item[\normalfont{Example:}] The Crustaceanator zaps Death Boy -- who is sitting 12 inches away -- with a microwave beam, even though Death Boy is behind the cover of a picture frame. During his next turn, Death Boy leaves his covered position and charges at the Crustaceanator, blazing away with his pointer. The Crustaceanator weighs his options and decides he cannot lose a turn to resetting the microwave beam so he backpedals, once again putting 12 inches between him and Death Boy. The Crustaceanator fires and roasts Death Boy to a crisp.
 \end{description}

\paragraph*{Sonic Blaster}
\label{sec:SonicBlaster}
\hspace*{0px}\\(\textit{\$1.50, stuns opponent, Weight 1})\\
A stun weapon that uses a concentrated blast of sound waves. Any crab hit by a sonic blaster must roll against Endurance or be stunned for one turn. A sonic blaster and a pickle spear can be a lethal combination if a crab is very fast. Very very tiny earplugs are included for free.

\subsection*{Armor}
\label{sec:Armor}

\paragraph*{Regular Armor}
\label{sec:RegularArmor}
\hspace*{0px}\\(\textit{\$0.50 per 10 points, Weight 1 per 10 points})\\
Regular armor is homemade armor made of shell, metal, or Bondo that the owner sticks to his/her crab's shell. Armor is purchased in 10-point increments. Each 10 points of additional armor adds one point of Weight.

\paragraph*{Kravlar Armor}
\label{sec:KravlarArmor}
\hspace*{0px}\\(\textit{\$1.50 per 20 points, Weight 1 per 20 points})\\
High-impact plastic armor that is sprayed on to the crab's existing shell, Kravlar grants the crab 20 points of armor per point of weight, but is really expensive.

\subsection*{Other Gear}
\label{sec:OtherGear}

\paragraph*{HalfTrack}
\label{sec:HalfTrack}
\hspace*{0px}\\(\textit{\$3.00})\\
The perfect gift for the crab who has everything, a HalfTrack is a set of tank-like treads that provide a boost to crabs who have too much equipment. The HalfTrack is heavy, but it does not add to the crab's Weight score, because the shell rests on the HalfTrack. A crab with a HalfTrack has a Movement score of 2, no matter how much weight it is carrying (up to Maximum Shell Capacity). A good accessory for a crab whose Weight is at or near Maximum Shell Capacity. A crab with a HalfTrack cannot jump or use a jetpak.

\paragraph*{Chia Pot}
\label{sec:ChiaPot}
\hspace*{0px}\\(\textit{\$1.00 per dose, no weight, one-time use})\\
One of the first off-the-books inventions to emerge at AssHoTech, Chia Pot is a dwarf variety of marijuana that can be grown, surreptitiously, right in the office. Chia Pot seeds are generally sold in a grapefruit-sized wad wrapped in pantyhose. Inside the wad, along with the seeds, is a growing medium and nutrients -- everything you need to grow Chia Pot. To get your farm started, all you have to do is soak the ball in water, place the ball on your windowsill, and wait for the seeds to sprout. To disguise the fact that they are growing an illegal plant, AssHoTech workers often put their Chia Pot balls in playful containers resembling cartoon characters, celebrities (Einstein and Don King are popular) or political figures.

Harvesting and preparing Chia Pot for human consumption is a painstaking process, sort of like threading a needle and searching a kid's head for lice at the same time. THC levels in the plant are low to nonexistent (the plant was engineered from the non-recreational, research-grade pot that can be legally grown in labs.) However, when hermit crabs chew the leaves, they often get a mild high. Think of teenaged Somali gunmen chewing khat leaves, and you get the idea.

A single dose of Chia Pot is expensive, and can be used only once. However, chewing the drug will give the crab a temporary +1 on Strength, Endurance and Dexterity and all scores derived from those abilities for the duration of the match. A crab can be given more than one dose, for a cumulative effect, but there is a risk of death by overdose. If a crab consumes more than one dose per turn, the player should roll one die: if the roll is lower than the number of doses, the crustacean has gone the way of Jimi Hendrix and Jim Morrison.

\paragraph*{JetPak}
\label{sec:JetPak}
\hspace*{0px}\\(\textit{\$2.50 Weight 3})\\
A deadly advantage for the crab-fighter who can afford it, the jetpak contains tiny rockets that allow a crab to fly rapidly through the air, shell and all. A crab can move 15 inches per turn with a jetpack, with no need to move around obstructions (coffee cups, etc.) The crab can jump straight up 15 inches, move laterally 15 inches, or any combination of movement (10 inches up and five inches forward, for instance). While moving by jetpak, a crab can shoot with a ``death from above'' advantage (+1 To Hit, \textminus1 to be hit by his opponent). If a crab is thrown off a desk or other object, he can avoid damage from the fall by firing his jetpak. A jetpak can be used only three times per combat, but can be recharged at no cost between combats.

\paragraph*{Crappy JetPak}
\label{sec:CrappyJetPak}
\hspace*{0px}\\(\textit{\$0.50, Weight 2, disposable})\\
This is a one-time jetpak fashioned from two Bottle Rockets. Cheap and unreliable, the system can nevertheless sometimes give that critical boost needed in combat. The user of a Crappy JetPak can move 15 inches in JetPak fashion, but can only do so once. The user discards the Crappy JetPak after use -- it's held on with Scotch\textsuperscript{\textregistered} tape -- and loses the Weight associated with it. Crappy JetPaks often misfire or tun out to be duds. Each time you use one, roll one die. If the result is a 1, the jetpak does not fire. If you roll a six, the jetpak fires but explodes at the end of flight, doing 2D6 damage to the crab's shell.

\section*{The World of Crab-fighting}
\label{sec:TheWorldofCrabfighting}

Kang was thrilled. After months of fighting his way to the top of the crab-fighting rings in Biomedical, Nanotech and IT, he had an experienced crab weighted down with deadly equipment. And here he was in Public Relations -- where the meanest bouts went down -- to spend his lunch break fighting for the championship of the entire wing.

There was more than money at stake. Ever since his crab, Crustaceous Clay, started winning bouts, Kang had been on the rise around the office. Naturally a wallflower, Kang was becoming a microcelebrity at AssHoTech.

Kang watched Gleefully as Clay faced off against the Desk Warrior, a small, scarred crab equipped with a flamethrower, a pointer and a crappy jetpak. Half the workers in the PR office crowded around the desk, debating odds and slapping their money down on the desk.

The fight began -- but disaster quickly unfolded. For his opening move, the Desk Warrior lit up his jets, hoping to close the space between him and Clay. But the crappy jetpak lived up to its name, misfiring with a crackling explosion. Workers ducked as the poor crab flew past them screaming, his body now a flaming ball of sulfur and charcoal. He landed in a wastepaper basket nearby. The basket went up in a sudden, cinematic gout of flame.

Kang rushed up with a fire extinguisher, but he couldn't get the little safety pin out. His heart pounded in his head -- then it sank as the sprinkler system came on, dousing the entire office.

Kang knew it was the end. With the fire, there would surely be an incident report. He would be fired for crab-fighting. And then what? How would he keep his visa? How would he find work elsewhere? How could he go home in shame?

\subsection*{A Secret Culture}
\label{sec:ASecretCulture}
Wargames and role-playing games used to have a certain dissident panache. Back in the early 1980s -- when the mainstream culture knew RPGs only through Pat Robertson, Geraldo Rivera and films like Mazes and Monsters -- to openly admit to gaming was to open yourself to accusations of Satanism and criminal insanity. Nobody got burned at the stake, but if you were in your teens at the time, the anti-RPG backlash certainly had a Salem-like feel. Some people went underground and gamed in secret. Others kept the faith, running their games in what, retrospectively, seems like a rather brave act of defiance.

Well, that was a long time ago. The dungeoneers of the 1980s are the working stiffs of the 21st-Century. And in geek-heavy work environments (academia, IT, etc.), gripes about the D20 system have joined Monday Night Football as a primary topic of watercooler conversation.

\warcrabs\hspace*{4pt}is an attempt to bring back some of the illicit majesty of the old wargaming culture. The game is designed to be played on the sly, right in your office. Like workers at AssHoTech, \warcrabs\hspace*{4pt}players have the most fun when the game becomes a secret hobby, a cryptic way of whiling away a break at work. The rules below explain the basics of AssHoTech's crab-fighting culture and show you how to simulate that culture in your office.

\subsection*{The Rules of Crab-fighting}
\label{sec:TheRulesofCrabfighting}
At AssHoTech, the crab-fighting rings are governed by just a few rules of honor. AssHoTech workers may be enlightened Pastafarian science fiction fans with advanced degrees, but they regard these rules with a deep and almost religious reverence.

The rules:

\begin{enumerate}
 \item Never talk about crab-fighting.
 \item NEVER TALK ABOUT CRAB-FIGHTING.
 \item Never spend more money than you've won.
 \item It's all about unintended consequences.
\end{enumerate}

Rule \#4 is perhaps the one most in need of explanation. When you read the introduction to this game, you may have asked yourself why prosperous, successful college graduates would risk their careers to watch a few crustaceans beat the shit out of each other. Or you may have asked yourself, ``How can I get in on this thing?'' The world seems to be divided between safe, sensible people and people who like to take risks and see what happens. Crab-fighters are definitely of the latter category. They recognize that the risk of career disaster is part of the fun. They accept, with an almost fatalistic surety, that if they are fired for crab-fighting, it was meant to be.

Indeed, as the people at the Employee Assistance Program keep reminding us, AssHoTech workers may subconsciously crave a downfall. After all, the world is rapidly going to hell in a handbasket, but these techies are still able to party like it's 1999. Maybe they just can't take the cognitive dissonance of being wealthy in a deteriorating world.

\subsection*{Making Money}
\label{sec:MakingMoney}
At AssHoTech there is a small underground economy surrounding the crab-fighting rings. The amounts of money in this economy are very small -- but because workers are honor-bound to spend only the money they win in the game, crab-fighting money isn't like other money. It's almost sacred.

In AssHoTech's crab-fighting matches, prizes are determined by a complex network of bookkeepers. To avoid all the complexities of bookkeeping, \warcrabs\hspace*{4pt}has a simple system for simulating the prize system in crab-fighting.

Each battle comes with a prize of \$0.10\texttimes1D6 for each opposing crab in the match (it is possible to have a Grand Melee featuring many crabs and a large prize).

Also, the last crab left standing wins all the money remaining in the account of the crabs he defeats. (A good incentive to buy all the equipment you can afford before each combat.)

If your crab is killed, you have to start over with a new crab and \$1.98 in cash.

\subsection*{Experience Points}
\label{sec:ExperiencePoints}
For each crab you defeat in combat, you get one Experience Point, which you can use to increase your crab's abilities. Each experience point can be used to increase your crab's Strength, Dexterity or Endurance by one point. Please note that each of these abilities has a maximum value of 10. If you max out all your abilities, you can use experience points to increase your base number of Hit Points, giving you a Hit Point value that exceeds your crab's Strength.

\subsection*{Simulating the Witch Hunt}
\label{sec:SimulatingtheWitchHunt}
At AssHoTech, getting caught crab-fighting is a career-ender -- but fortunately, there are many people who are in on the secret and are willing to cover for you.

That explains why crab-fighters are able to equip their warcrabs with fireworks and flaming weapons, and use them in combat, without always getting caught and losing their jobs. Obviously, the sound of firecrackers in a modern workplace would normally bring SWAT teams and news crews running. But at AssHoTech, pyrotechnics are accepted as part of the background noise. When they hear explosions from another office, AssHoTech workers just cough a lot and say, ``Oh, that noise? It was probably just that copier acting up again.''

But sometimes things happen that are too big to cover up. There are mishaps -- uncontrollable fires, injuries to workers and damage to equipment -- that can only lead to an investigation. And sometimes high-level managers, perhaps even T. Fred himself, just walk right into an office and catch a ring of crab-fighters.

We thought about making rules to govern mishaps, but this got too complicated and slowed down game play. So we came up with a simple rule to govern getting caught. \textbf{If you're ever forced to explain what you're doing to an outsider, consider yourself caught -- and fired.}

Here's an example of what I'm talking about.

\begin{quote}
Ethan and Leroy are locked in heated combat in Leroy's cubicle during a coffee break when Michelle approaches.
\begin{description}
 \item[MICHELLE:]Hey, what are you guys doing?
 \item[ETHAN:]Ah, nothing. Just talking shit. Nothing really.
 \item[LEROY:](Smiles and says nothing. He really, really likes Michelle.)
 \item[MICHELLE:]I see dice over there. Don't tell me you guys are getting up a D\&D campaign. Middle school is over!
 \item[ETHAN:]Busted. Once a dungeon master, always a dungeon master.
 \item[MICHELLE:]You guys... (Shakes her head. She smiles warmly, but with a hint of pity in her eyes.)
 \item[LEROY:]Oh, come on, Ethan. Just tell her the truth. Michelle, we're not playing D\&D. We're playing this cool game called \warcrabs. It's sort of a corporate-satire thing. Very hip. Didn't you read about it in Wired?
 \end{description}
\end{quote}

In this scenario. Leroy truly is busted. As is Ethan. He has broken the first rule of crab-fighting, which is not to talk about crab-fighting. In the game world, he and Ethan are both caught and fired.

Here's an alternate scenario:

\begin{quote}
\begin{description}
 \item[MICHELLE:]I see dice over there. Are you guys getting up another game of D\&D?
 \item[ETHAN:]Hey, look, don't tell anybody, but... we're playing craps.
 \item[MICHELLE:]No way, how old school.
 \item[ETHAN:]It was Leroy's idea. When he was working at the Shanghai office, he used to always see working-class Chinese guys playing craps in the street. He started joining in and kind of developed a taste for it. He's pretty good.
 \item[MICHELLE:]Whoa. Cool. I don't even know how to play craps. Can you show me, Leroy?
 \end{description}
\end{quote}

Now that's how it's supposed to be done. Leroy and Ethan stay in the game, and Leroy might even get the girl.

If you do get caught, you have to discard (see the Rogue Crabs rules below) your current crab and start over again with a new crab. In game terms, you are now a new worker, just joining the crab-fighting culture.

This rule applies only if you're ``caught'' while in an actual crab-fighting session. You are free to talk about crab-fighting between battles, and even to recruit new players to the game. And if you are ``caught'' crab-fighting by another \warcrabs\hspace*{4pt}player, it doesn't count. You can keep your crab and go on like normal. A good reason to recruit your entire office into a crab-fighting ring!

\subsection*{Rogue Crabs and MechaCrabs}
\label{sec:RogueCrabsandMechaCrabs}

When AssHoTech management banned crab-fighting, they neglected to think about one important ramification of busting up crab-fights. Crabs don't necessarily like leading lives of gladiatorial combat, so there is always the risk they will run away. When workers are scrambling to hide an ongoing match, crabs often take the opportunity to make a run for it. Sometimes workers intentionally let their crabs go during a bust, so there will be less evidence of wrongdoing. And finally, fired employees often maliciously release their crabs to get revenge on the company. As a result, AssHoTech's campus is crawling with ``rogue'' crabs.

No one has done a definitive, peer-reviewed study on the intelligence of fighting crabs. However, according to AssHoTech's biotech guys, the average fighting crab's brain-to-body-mass ratio should give it the intelligence of about a seven-year-old kid -- a cold-blooded, khat-chewing, video-game-addled seven-year-old kid. So it should not be too surprising that AssHoTech's rogue crabs have exterminated every competing vermin in the building -- rats, mice, roaches and all. They manage to feed themselves quite well by raiding the cafeteria and even conducting daring, daylight raids on workers' bag lunches. They've also managed to keep themselves fairly well-armed by plundering the office armories of workers in crab-fighting rings.

There is some evidence that the crabs are even building their own fortifications in the unseen spaces of the AssHoTech campus. The discovery of a large dirt-dauber-like nest in the sub-basement of Building D caused quite a stir in the company. No one knows where the crabs got their building material (contrary to popular opinion, the nest was not made out of Liquid Paper), but given the weapons caches inside, it was clear that crabs once lived here. Contractors later found 25 well-cultivated heads of Chia Pot growing in a little-noticed roof gutter, but it is still not clear whether this was grown by crabs or humans.

After studying the problem and attempting various traditional solutions (fumigation, traps, fly paper etc.) AssHoTech decided to approach the problem in true biotech fashion. Rather than wage an asymmetric war against rogue crabs, AssHoTech engineered and mass produced a tiny robot -- a MechaCrab -- that could fight and eliminate the crab on its own level. Unfortunately, the company did not tap into the wisdom of actual crab-fighters in launching the project, and as a result, the MechaCrab is only a fair-to-middling fighter.

Both rogue crabs and MechaCrabs present a danger to crab-fighters. Chance encounters with either kind of enemy crab will lead to a sudden, unexpected and potentially deadly combat situation.

\subsubsection*{Simulating Rogues and MechaCrabs}
\label{sec:SimulatingRoguesandMechaCrabs}

Once you get a good crab-fighting ring going (say, four to six people) you may want to appoint a RogueMaster.

The RogueMaster's job is fairly simple. Whenever someone loses their crab due to being ``caught'' in a crab-fight, the RogueMaster takes over that player's crab (it might be a good idea to keep a file folder for all the character sheets you collect). That crab can later return for a surprise attack on any player.

How does a RougeMaster launch a surprise attack? Just fold the rogue crab's character sheet in half (or, if you want to be fancy, fold it into an origami crab) and write ``ROGUE'' on it in bold letters. Leave it in a common area (say, on top of a copier). The first player who encounters the crab must fight it, right where it was found. The RogueMaster gets to control the rogue crab throughout the combat.

If you do not have any rogue crabs (or want to give your players an easy challenge) you can surprise them with a MechaCrab. The MechaCrab has very simple characteristics, as listed below:

\subsubsection*{MechaCrab}
\label{sec:MechaCrab}

Strength: 5\newline{}Dexterity: 5\newline\bigskip{}Endurance: 5\newline{}Armor: 10\newline{}Movement: 5\newline\bigskip{}Hit Points: 5\newline{}Weapons: Pointer, 2 firecrackers

\begin{description}
 \item[Playing Fair --] The RogueMaster should be a fair-minded (but mischievous) person, and should not manage any crab except rogues and MechaCrabs. To avoid accusations of meanness or bias, Rogues should never place crabs on another person's desk or in any area that seems to target a specific individual. Common areas like stairwells are a good place to leave a crab. Little-used closets and storage rooms are even better. A rogue might lie in wait for months before someone chances across it.
 \item[Tactics --] A rogue will fight until the RogueMaster determines it is seriously damaged, and then it will try to retreat. A MechaCrab will battle to the death.
 \item[Experience Points and Prizes --] There is no monetary prize for fighting outside the crab-fighting ring. However, any time a crab survives rogue or MechaCrab combat, it acquires one Experience Point. Rogues also acquire one Experience Point for each combat they survive. MechaCrabs do not learn from past experience: they do not have either the software or the hardware for acquiring Experience Points.
 \end{description}

\section*{Holy War (Optional Rule)}
\label{sec:HolyWar(OptionalRule)}

In 2005, to protest the Kansas Board of Education's decision to introduce ``intelligent design'' in the classroom, graduate student Bobby Henderson wrote an open letter to the School Board demanding that they give equal classroom time to Flying Spaghetti Monsterism. Henderson outlined his belief of an invisible, supernatural creator called the Spaghetti Monster, whose existence could be neither proven nor dis-proven, and insisted that the Monster deserved a place in the curriculum along with the story of ``intelligent design.''

People didn't just get the joke -- they actually laughed. And they started ``worshiping'' the Spaghetti Monster. Across college campuses, the new religion, often known as ``Pastafarianism,'' spread like wildfire.

The new parody cult had a huge effect on AssHoTech. From 2005 onward, virtually all of AssHoTech's interns and new, under-25, employees have been self-described Pastafarians, a semi-private joke they find immensely satisfying.

For AssHoTech's Gen-X workers, however, it's a joke they've heard before. Virtually all of that generation have some affiliation with Ivan Stang's Church of the Subgenius -- which, they will be happy to tell you, is the ``original joke religion.'' Now that they are older (and have even come to resemble Subgenius deity Bob Dobbs), the Gen-Xers are unwilling to tolerate a bunch of whippersnappers who think they've invented religious parody out of whole cloth.

Thus began the Joke Jihad. It started when a Subgenius painted the slogan (``Love + Slack'') on the shell of his warcrab, which then proceeded to torture a new employee's crab to a miserable death in the ring. Later, a Subgenius player's crab was found murdered in a storeroom -- a death that would have been attributed to rogues, if not for the graffiti found at the site: ``There is no God but pasta, and Captain Mosey is its prophet.''

Since then, sectarian warfare has become part of the crab-fighting culture. New players are encouraged to declare themselves either Pastafarian or Subgenius, and they occasionally engage in gangland-style warfare.

If your group decides to play the holy war, each player should declare a religion. They will then be governed by the following rules:

\begin{enumerate}
 \item Players of the same religion can fight each other in the crab-fighting ring. However, in group combat, ``religious'' crabs are required to eliminate all crabs from the opposing religion before attacking their own co-religionists.
 \item The crab-fighting code prohibits one crab-fighter from attacking another directly, outside the bounds of a fair and formal crab-fight. However, individual fighters may pool their resources to launch plausibly-deniable surprise attacks on their opponents. For \$2.00 in-game cash, a player or group of players may buy a completely new, unnamed crab for use in a surprise, kamikaze attack on another crab. Any equipment for the kamikaze crab must be purchased at additional expense to the attackers.
 \item The attackers decide when and where the kamikaze will attack, but the RogueMaster controls the kamikaze crab during combat.
 \item A kamikaze crab will always fight to the death.
 \item The RogueMaster cannot reveal the identity of the attackers.
\end{enumerate}

\pagebreak

\section*{Frequently Anticipated Questions}
\label{sec:FrequentlyAnticipatedQuestions}

This is Version 1 of Warcrabs, so no one has submitted any questions about the game yet. However, we have anticipated a couple that people might ask. If you have more, send them to: dimolaoshi@yahoo.com.

\subsection*{What if I don't have a ruler that measures in inches?}
\label{sec:Faq1}

\warcrabs\hspace*{4pt}was invented in America, where we use crazy but poetic measurements like ``feet'' and ``pounds.'' If you are in a country that uses a more rational system, you have three options.

\begin{enumerate}
 \item Convert the figures. There are about 2.5 centimeters per inch, more or less. We're not mixing chemicals here, so you can fudge a little.
 \item Wait for America to invade your country. When our plans for global domination are complete, everyone will use English measurements. BWAH-HAH-HAH!
 \item Just go ahead and play using centimeters in place of inches. This will lead to a less acrobatic game -- sort of like watching high school basketball, as opposed to the NBA -- but it might work.
\end{enumerate}

\subsection*{What if I kill enough opponents to max out all my abilities (Strength 10, Dexterity 10, Endurance 10)? Won't this mess up the balance of the game?}
\label{sec:Faq2}

Don't worry. You won't live that long unless you are cheating. And if the game balance gets skewed, we're sure the RogueMaster can take care of that.

\subsection*{Can crabs drive little toy cars?}
\label{sec:Faq3}

If the car has an engine and a crab can fit inside, go for it.

\subsection*{Can I invent new equipment/rules for \textit{Warcrabs!}?}
\label{sec:Faq4}

Feel free to alter, improve, or completely rewrite the game, and even release your new
version to the public -- as long as you don't make money off the game, and as long as
you credit Di Mo as the author of the original game (see the Creative Commons
copyright below). Do please send the author your revisions, ideas and questions at
dimolaoshi@yahoo.com.

\section*{Disclaimer: Keep it in the fantasy world, okay?}
\label{sec:Disclaimer}

\warcrabs\hspace*{4pt}grew out of a conversation between the author and one of his role-playing friends. The gist of the conversation: that the hermit crab -- once advertised in comic books as the ``Crazy Crab'' -- would make a perfect RPG character since so much of its appearance and personal identity is based on the stuff it carries on its back. Of course, all crabs have shells, but hermit crabs steal shells from others, then move into new shells when they grow too big -- just like RPG characters moving from one suit of armor to the next. This conversation grew into the idea of ``battle crabs'' and finally \warcrabs.

No actual crabs were harmed in the making of this game, and we would like to keep it that way. \textit{Please} keep in mind that this game is satirical. It's a satire of the corporate world and of the unintended consequences of genetic engineering. It is not intended in any way to encourage people to pit crabs in battle against one another, if this is even possible. This author is a big believer in the pseudo-hippie ethos of the early gaming community -- where there was an unwritten rule that combat is fun on paper but fighting sucks in the real world. So \textit{please} be nice to real hermit crabs.

Let me put it another way. You can buy guns and armored cars and go autodueling in the desert with consenting adults if you want to. You can walk around mumbling to yourself about Gorvils and clutching your Crown Royal bag full of spells if you wish. But please don't hurt innocent animals.

\section*{T. Fred Hawkins: American Innovator}
\label{sec:TFredHawkins}

T. Fred Hawkins, founder and CEO of Associated Holistic Technologies, has been called ``the most innovative business thinker of our time'' (\textit{Publisher's Weekly}), ``a charismatic, shoot from the hip wheeler-dealer'' (\textit{Austin Statesman}) and ``Teddy Roosevelt with a Texas accent and big ears'' (\textit{Time}).

Born on a cotton farm in East Texas, Hawkins fast-talked his way into an entry-level position in locally-owned Grumm Oil, eventually rising to the top of the company and displacing founding CEO Ed Grumm. He rose to world fame after two Grumm Oil prospectors were arrested for allegedly acting as ``sex tourists'' in Yemen. To the chagrin of the Reagan administration, Hawkins hired a mercenary team to go in, guns blazing, to rescue them.

Hawkins used his wealth and newfound fame to launch a third-party bid for the Presidency in 1984. While he picked up only 2 percent of the vote, Hawkins became and international sensation -- and killed the Democrats' chances in the election -- when he called Walter Mondale a ``pantywaist'' in a nationally televised debate.

When his profits were threatened during the oil glut of the 1980s, Hawkins saw the writing on the wall. By then Hawkins had controlling interest of Grumm -- and he sold the company to get into the fledgling business of cloned PCs. Soon he was again the talk of the nation, when his business/self-help book \textit{Four Habits of Bad Ass Motherfuckers} sold 19 million copies. Millions of American workers have since gone through intense training sessions based on this business classic, leading \textit{Newsweek} to declare that ``nobody has contributed more to the cowboy image of the American businessman than T. Fred Hawkins.''

Hawkins is now leader of AssHoTech, the nation's leading provider of technologies that combine traditional engineering and biotechnology. He prides himself on having a one-on-one, personal relationship with workers at all levels of the company.

\pagebreak

\subsection*{From T. Fred Hawkins' Four Habits of Bad Ass Motherfuckers}
\label{sec:4Habits}

\begin{enumerate}
 \item \textit{Point a lot.} Point right at people.
 \item \textit{Know their names.} When speaking, use the listener's name often, especially when pointing.
 \item \textit{Self-reference.} Find ways to bring the strong points of your resume into the conversation, whether they seem to fit or not. (``When I was on the hostage rescue mission in Yemen, the ex-seals and I listened to a lot of baseball games on Voice of America. And I can tell you there's no way Pete Rose threw any of those games.'')
 \item \textit{You've got a gun in your pocket.} Use it. i.e. when you're speaking in public, don't imagine them all in their underwear -- it's too distracting. Imagine that you've got a pistol in your pocket (or actually bring a pistol). Somebody needs to bring order here, and you're the only one with a gun. So walk with some swagger.
\end{enumerate}

\section*{About the Author}
\label{sec:AbouttheAuthor}

\textbf{Di Mo} (rhymes with ``Nemo'') is the pen name of a writer and game geek who lives in Gainesville, Florida. His work has appeared in \textit{Star Frontiersman}, and his novel \textit{Iskander} is currently seeking a publisher. He can be reached at \dimoEmail.

\textbf{kdnewton} is responsible for this revised edition of the \warcrabs\hspace*{4pt}manual. He can be reached at \kyleEmail. Any and all credit toward \warcrabs, at least within the context of this edition, is to \href{mailto:mailto:dimolaoshi@yahoo.com}{Di Mo} as the original author. Any grammar/spelling corrections should be directed to \href{mailto:kyle@n3wt0n.com}{kdnewton}.

This revision of the \warcrabs\hspace*{4pt}manual was originally hosted at\newline\url{http://n3wt0n.com/pmwiki/index.php?n=TheLab.Warcrabs}

\section*{Copyright}
\label{sec:Copyright}

%% You'll notice *two* identical links in the following paragraph.
%% The compiled PDF displays the text in a link properly, breaking where it should. However
%% the compiled DVI (and the compiled PS) will run the link-text off the page.
%% So we break the link ourselves.
This work is licensed under a \href{http://creativecommons.org/licenses/by-sa/3.0/}{Creative Commons Attribution-Share Alike 3.0} \href{http://creativecommons.org/licenses/by-sa/3.0/}{Unported License}. You're free to distribute it for free, and to alter it, as long as you credit \href{mailto:mailto:dimolaoshi@yahoo.com}{Di Mo} as the original author.

It would also be nice to include the e-mail address of the original author (\dimoEmail) in all future versions so he can collect feedback from players.

\section*{...and another thing...}
\label{sec:andanotherthing}

This game depicts a fictional situation, and is a work of satire. Any resemblance between the characters herein and any actual people or organizations, with other than satirical intent, is entirely coincidental.

\pagebreak

\label{sec:CharacterSheet}

\small{Crab Character Sheet}

\huge\textbf{Crab Name:}

%Need this here to set the vertical spacing between the Crab Name and the table
\bigskip

\normalsize
\textbf{
 % use packages: array
\begin{tabular*}{0.90\textwidth}{@{\extracolsep{\fill}} r r }
 Strength: \underline{\hspace{1.5cm}} & Optimum Shell Capacity: \underline{\hspace{1.5cm}} \\ 
 Dexterity: \underline{\hspace{1.5cm}} & Maximum Shell Capacity: \underline{\hspace{1.5cm}} \\ 
 Endurance: \underline{\hspace{1.5cm}} & Movement: \underline{\hspace{1.5cm}} \\ 
 \medskip & Weight: \underline{\hspace{1.5cm}} \\ 
 \bigskip \\ 
 Hit Points: \underline{\hspace{1.5cm}} \\ 
 Armor: \underline{\hspace{1.5cm}} \\ 
 \bigskip \\ 
 Money: \underline{\hspace{1.5cm}}
 \end{tabular*}
}

\bigskip

\LARGE
\textbf{Equipment:}

\normalsize
\pagebreak

\section*{Charts \& Tables}
\label{sec:ChartsTables}

\subsection*{Figuring Crab Stats:}
\label{sec:FiguringCrabStats}

\begin{description}
 \item[Optimum Shell Capacity:] Same as the crab's Strength.
 \item[Maximum Shell Capacity:] The crabs Dexterity and Strength combined.
 \item[Movement:] Same as Dexterity except when Weight is greater than Strength. In that case it is (Dexterity minus (Weight minus Strength)).
 \end{description}

\subsection*{To Hit Modifiers:}
\label{sec:ToHitModifiers}

% use packages: array
\begin{tabular}{ | l | l | r | }
\hline
\textbf{Range} & \textbf{Distance} & \textbf{Modifier} \\ \hline
Point Blank & Less than 2\inch & +2 \\ \hline
Short & 2\inch to 6\inch & +1 \\ \hline
Medium & 6\inch to 12\inch & 0 \\ \hline
Long & 12\inch to 18\inch & \textminus1 \\ \hline
Extra Long & 18\inch or more & \textminus2 \\
\hline
\end{tabular}\newline Note: To Hit modifiers apply only to ranged weapons.

\subsection*{Equipment:}
\label{sec:EquipmentTables}

% use packages: array
\begin{tabular}{ l l c c l l }
\textit{Item} & \textit{Cost} & \textit{Weight} & \textit{To Hit} & \textit{Damage} & \textit{Max Range} \\ 
\hspace{2px} \\ %not looking for a \smallskip because that is too big
Armor (10) 	    & \$0.50   & 1     & \dots 	     & \dots & \dots \\ 
Bottle Rocket 	    & \$0.30   & 1     & \textminus1 & 2D6   & \dots \\ 
Chia Pot 	    & \$1/dose & 0     & \dots 	     & \dots & \dots \\ 
Crappy JetPak 	    & \$0.50   & 2     & \dots 	     & \dots & \dots \\ 
Firecracker 	    & \$0.23   & 1     & \dots 	     & 2D6   & Strength \texttimes 2 \\ 
Flamethrower 	    & \$2.00   & 2     & +2 	     & 2D6   & 10\textacutedbl \\ 
HalfTrack 	    & \$3.00   & \dots & \dots 	     & \dots & \dots \\ 
JetPak 		    & \$2.50   & 3     & \dots 	     & \dots & \dots \\ 
Kravlar Armor (20)  & \$1.50   & 1     & \dots 	     & \dots & \dots \\ 
MBRLS 		    & \$2.25   & 5     & +2 	     & 3D6   & \dots \\ 
Microwave Gun 	    & \$2.50   & 3     & \dots 	     & 2D6   & \dots \\ 
Pickle Spear 	    & \$0.20   & 1     & \dots 	     & 1D6   & 1\textacutedbl \\ 
Pointer 	    & \$2.00   & 1     & \dots 	     & 1D6   & \dots \\ 
Sonic Blaster       & \$1.50   & 1     & \dots 	     & stun  & \dots \\ 
Vorpal Pickle Spear & \$1.00   & 1     & \dots 	     & 1D6+2 & 1\textacutedbl
\end{tabular}

\end{document}
